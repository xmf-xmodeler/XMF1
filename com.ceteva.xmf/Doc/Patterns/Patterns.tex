\documentclass{article}
\usepackage{fancyhdr}

\pagestyle{fancy}
\lhead{}
\chead{Patterns in XMF}
\rhead{\thepage}
\cfoot{(c) 2004 Ceteva Ltd.}

\title{Patterns in XMF}

\author{Ceteva Ltd.}

\begin{document}

\maketitle

\section{Introduction}

XMF is an environment for domain specific language development. The basic
XMF environment is supplied with a language, called XOCL, that is based on
the OMG standard OCL language. XOCL is defined using the XMF language
development technologies and adds a number of features to OCL including:
higher order functions, syntax classes, meta-circularity, slot update, 
object creation, threads and pattern matching. The purpose of this document
is to describe the pattern matching facilities in XOCL.

XOCL aims to support the rapid development of applications using a rich
collection of language features. XOCL abstracts away from the details of 
{\em how} data is accessed and transformed by providing {\em pattern matching}
language features. Patterns provide an information-centric focus of
attention by emphasising {\em what} is being accessed and transformed rather
than {\em how} the data is navigated and modified. In this sense, XOCL provides
an abstraction layer that increases productivity and reduces the likelihood
of low-level programming errors.

\section{Patterns}

A pattern is matched against a value. The pattern match may succeed or fail
in a given matching context. A matching context keeps track of any variable
bindings generated by the match and maintains choice points for backtracking
if the current match fails.

Pattern matching can be viewed as being performed by a {\em pattern matching
machine} that maintains the current pattern matching context as its state. 
The engine state consists of a stack of patterns to be matched against a 
stack of values, a collection of variable bindings and a stack of choice points.
A choice point is a machine state. At any given time there is a pattern at
the head of the pattern stack and a value at the head of the value stack.
The machine executes by performing state transitions driven by the head of
the pattern stack: if the outer structure of the pattern matches that of the
value at the head of the value stack then:
\begin{itemize}
\item 0 or more values are bound.
\item 0 or more choice points are added to the choice point stack.
\item 0 or more component patterns are pushed onto the pattern stack.
\item 0 or more component values are pushed onto the value stack.
\end{itemize}
If the machine fails to match the pattern and value at the head of the 
respective stacks then the most recently created choice point is popped
and becomes the new machine state.
Execution continues until either the pattern stack is exhausted or the
machine fails when the choice stack is empty.

The rest of this section describes the different categories of pattern.
The semantics of matching are defined informally in terms of a general
description and example definitions involving the pattern.

\begin{description}

\item[Variables]
A variable pattern consists of a name, optionally another pattern and optionally a type.
The simplest form of variable pattern is just a name, for example, the formal parameter
{\tt x} is a variable pattern:
\begin{verbatim}
let add1 = @Operation(x) x + 1 end in ...
\end{verbatim}
Matching a simple variable pattern such as that shown above always succeeds and causes
the name to be bound to the corresponding value. A variable may be qualified with a type
declaration:
\begin{verbatim}
let add1 = @Operation(x:Integer) x + 1 end in ...
\end{verbatim}
which has no effect on pattern matching. A variable may be qualified with a pattern as in
{\tt x = <Pattern>} where the pattern must occur before any type declaration. Such a 
qualified variable matches a value when the pattern also matches the value. Any variables 
in the pattern {\em and} {\tt x} are bound in the process.

\item[Constants]
A constant pattern is either a string, an integer, a boolean or an expression 
(in the case of an expression the pattern consists of [ followed by an expression
followed by ]). A constant pattern matches a value when the values is equal to the 
constant (in the case of an expression the matching process evaluates the expression 
each time the match occurs). For example:
\begin{verbatim}
let fourArgs = @Operation(1,true,"three",x = [2 + 2]) x end in ...
\end{verbatim}
is an operation that succeeds in the case:
\begin{verbatim}
fourArgs(1,true,"three",4)
\end{verbatim}
and returns {\tt 4}.

\item[Sequences]
A sequence pattern consists of either a pair of patterns or a sequence of patterns. 
In the case of a pair:
\begin{verbatim}
let head = @Operation(Seq{head | tail}) head end in ...
\end{verbatim}
the pattern matches a non-empty sequence whose head must match the head pattern and
whose tail must match the tail pattern. In the case of a sequence of patterns:
\begin{verbatim}
let add3 = @Operation(Seq{x,y,z}) x + y + z end in ...
\end{verbatim}
the pattern matches a sequence of exactly the same size where each element matches
the corresponding pattern.

\item[Constructors]
A constructor pattern matches an object. A constructor pattern may be either a
by-order-of-arguments constructor pattern (or BOA-constructor pattern) or a 
keyword constructor pattern.

A BOA-constructor pattern is linked with the constructors of a class. It has the
form:
\begin{verbatim}
let flatten = @Operation(C(x,y,z)) Seq{x,y,z} end in ...
\end{verbatim}
where the class {\tt C} must define a 3-argument constructor. A BOA-constructor
pattern matches an object when the object is an instance of the class (here {\tt C} but
in general defined using a path) and when the object's slot values identified by the
constructor of the class with the appropriate arity match the corresponding
sub-patterns (here {\tt x}, {\tt y} and {\tt z}).

A keyword constructor pattern has the form:
\begin{verbatim}
let flatten = @Operation(C[name=y,age=x,address=y]) Seq{x,y,z} end in ...
\end{verbatim}
where the names of the slots are explicitly defined in any order (and may be repeated).
Such a pattern matches an object when it is an instance of the given class and when
the values of the named slots match the appropriate sub-patterns.

\item[Conditions]
A conditional pattern consists of a pattern and a predicate expression. It matches a
value when the value matches the sub-pattern and when the expression evaluates
to true in the resulting variable context. For example:
\begin{verbatim}
let repeat = @Operation(Seq{x,y} when x = y) Seq{x} end in ...
\end{verbatim}
Note that the above example will fail (and probably throw an error depending on the context)
if it is supplied with a pair whose values are different.

\item[Sets]
Set patterns consist of an element pattern and a residual pattern. A set matches a
pattern when an element can be chosen that matches the element pattern and where the
rest of the set matches the residual pattern. For example:
\begin{verbatim}
let choose = @Operation(S->including(x)) x end in ...
\end{verbatim}
which matches any non-empty set and selects a value from it at random. 

Set patterns introduce choice into the current context because often there is more
than one way to choose a value from the set that matches the element pattern. For 
example:
\begin{verbatim}
let chooseBigger = @Operation(S->including(x),y where x > y) x end in ...
\end{verbatim}
Pattern matching in {\tt chooseBigger}, for example:
\begin{verbatim}
chooseBigger(Set{1,2,3},2)
\end{verbatim}
starts by selecting an element and binding it to {\tt x} and binding {\tt S} 
to the rest. In this case suppose that {\tt x = 1} and {\tt S = Set\{2,3\}}.
The pattern {\tt y} matches and binds {\tt 2} and then the condition is applied. 
At this point, in general, there may be choices left in the context due to there 
being more than one element in the set supplied as the first parameter. If the 
condition {\tt x > y} fails then the matching process jumps to the most recent 
choice point (which in this cases causes the next element in the set to be chosen 
and bound to {\tt x}). Suppose that {\tt 3} is chosen this time; the condition
is satisfied and the call returns {\tt 3}.

The following is an example that sorts a set of integers into descending order:
\begin{verbatim}
context Root
  @Operation sort(S)
    @Case S of
      Set{} do Seq{} end
      S->including(x) 
          when S->forAll(y | y <= x) 
        do Seq{x | Q} where Q = sort(S) 
      end
    end
  end
\end{verbatim}

\item[Sequences]
Sequence patterns use the infix {\tt +} operator to combine two patterns
that match against two sub-sequences. For example the following operation
removes a sequence of {\tt 0}'s occurring in a sequence:
\begin{verbatim}
context Root
  @Operation remove0s(x) 
    @Case x of 
      (S1 when S1->forAll(x | x <> 0)) + 
        (S2 when S2->forAll(x | x = 0)) + 
        (S3 when S3->forAll(x | x <> 0)) 
      do S1 + S3 
      end 
    end
  end
\end{verbatim}

\item[Syntax]
Syntax patterns consist of expressions within quasi-quotes {\tt [|} and {\tt |]}.
The quotes are a short-hand for writing out the equivalent constructor
patterns. Syntax patterns provide a powerful way of constructing syntax
mappings where the pattern is defined in terms of concrete syntax rather
than the equivalent abstract syntax structures.

Consider an operation that extracts the body of a {\tt let} expression:
\begin{verbatim}
context Root
  @Operation getBody([| let x = value in body end |]) 
    body 
  end
\end{verbatim}
Unfortunately, this willnot work as you may expect since the syntax pattern
states that the operation expects to be supplied with a {\tt let} expression
that consists of exactly one binding and where the body of the expression
is the variable whose name is {\tt body}. We wish to place patterns within the
syntax construct that match against specific elements of the abstract
syntax structure. To do this we use pattern-unquotes:
\begin{verbatim}
context Root
  @Operation getBody([| let <| bindings |> in <| body |> end |]) 
    body 
  end
\end{verbatim}
Within a syntax pattern the unquotes {\tt <|} and {\tt |>} are used to
surround patterns that are to be matched against the abstract syntax
structures occurring at that point in the supplied expression. In the example
above, {\tt bindings} is bound to the sequence of bindings in the {\tt let}
and {\tt body} is bound to the body.

The following example shows an operation that calculates the free variables
occurring in an expression. The expression is limited to a small number
of XOCL expressions:
\begin{verbatim}
context Root
  @Operation FV(e)
    @Case e of
      [| let <| Seq{} |> in <| e |> end |] do 
        FV(e) 
      end
      [| let <| Seq{ ValueBinding(v,e1) | bs } |> 
         in <| e2 |> 
         end 
      |] do
        FV([| let <bs> in <e2> end|])->excluding(v)
      end
      [| if <| e1 |> 
         then <| e2 |> 
         else <| e3 |> 
         end 
      |] do
        FV(e1) + FV(e2) + FV(e3)
      end
      [| <| e1 |> = <| e2 |> |] do
        FV(e1) + FV(e2)
      end
      Var[name=n] do 
        Set{n} 
      end
    end
  end
\end{verbatim}
The following call:
\begin{verbatim}
FV[| let x = 10; y = 20 in if x = y then z else a end end |])
\end{verbatim}
produces the set {\tt Set\{a,z\}}.

\end{description}

\section{Pattern Contexts}

Patterns may be used in the following contexts:

\begin{description}

\item[Operation Parameters]
Each parameter in an operation definition is a pattern. Parameter patterns
are useful when defining an operation that must deconstruct one or more values
passed as arguments. Note that if the pattern match fails then the operation
invocation will raise an error.

Operations defined in the same class and with the same name are merged into a 
single operation in which each operation is tried in turn when the operation
is called via an instance of the class. Therefore in the following example:
\begin{verbatim}
@Class P
  @Operation f(Seq{}) 0 end
  @Operation f(Seq{x | t}) x + self.f(t) end
end
\end{verbatim}
an instance of {\tt P} has a single operation {\tt f} that adds up all the 
elements of a sequence.

\item[Case Arms]
A case expression consists of a number of arms each of which has a sequence
of patterns and an expression. A case expression dispatches on a sequence of
values and attempts to match them against the corresponding patterns in each
arm in turn. For example, suppose we want to calculate the set of duplicated
elements in a pair of sets:
\begin{verbatim}
context Root
  @Operation dups(s1,s2)
    @Case s1,s2 of
      s1->including(x),s2->including(y) when x = y do 
        Set{x} + dups(s1,s2) 
      end
      s1->including(x),s2 do 
        dups(s1,s2) 
      end
      s1,s2->including(y) do 
        dups(s1,s2) 
      end
      Set{},Set{} do 
        Set{} 
      end
    end
  end
\end{verbatim}

\item[Mapping Clauses]
Mappings contains clauses. A clause is a named case arm and therefore mappings
can contain patterns. The following mapping transforms a class to a sequence
of all the parent classes including itself:
\begin{verbatim}
  @Map AllClasses(EMOF::Class)->Seq(EMOF::Class)
    @Clause MapClass
      c = EMOF::Class[parents = P] do 
        Seq{c} + S where S = self(P) 
    end
    @Clause MapClasses
      P->including(p) do 
        self(p) + self(P) 
    end
    @Clause MapEmpty
      Set{} do 
      Seq{}
    end
  end
\end{verbatim}

\end{description}

\section{Syntax}

The following BNF grammar defines the language of patterns in XOCL. 
The following conventions have been used: terminals are surrounded by single
quotes; non-terminals start with a capital letter; non-terminals Integer,
String and Boolean are assumed; $*$ is used postfix to mean 0 or more 
repetitions; [ and ] surround optional elements.

{\small
\begin{verbatim}

Pattern ::= Add ('->including(' Pattern ')')* [ 'when' Exp ].

Add ::= Atom [ '+' Add ].

Atom ::= Varp | Constp | Cnstrp | Seqp | Setp | Syntaxp | '(' Pattern ')'.

Varp ::= Name [ '=' Pattern ] [ ':' Type ]

Constp ::= Integer | String | Boolean | '[' Exp ']'

Cnstrp ::= BOAp | Keyp

BOAp ::= Path '(' [ Pattern (',' Pattern)* ] ')'

Keyp ::= Path '[' [ Name '=' Pattern (',' Name '=' Pattern ] ']' 

Seqp ::= 'Seq{' [ Pattern (',' Pattern)* ] '}' | 'Seq{' Pattern '|' Pattern '}'

Setp ::= 'Set{}'

Syntaxp ::= '[|' Exp '|]'.

Path ::= Name ('::' Name)*.

\end{verbatim}}

\end{document}