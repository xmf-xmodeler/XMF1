\documentclass{article}
\usepackage{fancyhdr}

\pagestyle{fancy}
\lhead{}
\chead{The XMF Meta Object protocol}
\rhead{\thepage}
\cfoot{(c) 2004 Ceteva Ltd.}

\title{The XMF Meta Object protocol}

\author{Ceteva Ltd.}

\begin{document}

\maketitle

\section{Introduction}

XMF is an environment for language design and deployment. Languages control the structure and
behaviour of the values that they denote. In this sense, language design is a {\em meta-activity}
and requires a meta-language that represents the structure and behaviour of the language components.

XMF provides a meta-circular object-oriented kernel language called XOCL. The meta-circular
property means that XOCL is defined completely in itself. This property validates XOCL as a
meta-language. Object-orientation provides a basis for application extension and reuse through
inheritance and modularity through encapsulation. 

XOCL is both meta-circular and object-oriented, it is suitable for language definition where 
languages can be easily constructed as modular extensions of the basic XOCL language. Instantiations
of XOCL are languages and extensions of XOCL are meta-languages.

All languages have key semantic features that can be represented as an interface in the definition
of the language. Consider a language with an operational semantics. In this case programs written
in the language may be viewed as controlling a machine that contains the state of the execution
at any given snapshot in time. The key semantic features of the language form the API of the
machine.

Given a language {\em L}, we would like to construct a new language that is {\em L}-like. If {\em L}
is defined using object-oriented principles then it is attractive to construct the new language as
an extension of {\em L} using inheritance. Syntax structures and values of the new language can be 
defined by extending the appropriate features of {\em L}. We would like to construct the semantics 
of the new language using the same approach. If we have constructed the semantics of {\em L} by
encapsulating the key features as an implementation of the API as described above, then the new
language semantics can be defined by inheriting and extending these operations as appropriate.

Where the semantics of a language has been constructed using object-oriented principles, the
resulting collection of classes and operations is referred to as a {\em meta-object-protocol}.
This document describes the XOCL MOP.

\section{Message Passing}

XOCL performs computation in terms of messages between elements. A message consists of a name and
some data. A message is sent from a source element to a target element. The target element receives the
message, performs appropriate computation and returns a result. Messages between elements are
synchronous: the source element halts computation and waits the return value from the target
element.

Message passing occurs when the source element performs an expression of the form: {\tt o.m(x,y,z,...)}
where {\tt o} is the target element, {\tt m} is the message name and {\tt x}, {\tt y}, {\tt z} etc. is
the data, or {\em parameters}, of the message.

Message passing is defined by the MOP component referred to as the {\em massage passing protocol}.
The protocol is defined by the meta-class of the target element and is called {\tt sendInstance}:
\begin{verbatim}
context Element
  @Operation send(message,args):Element
    self.of().sendInstance(self,message,args)
  end
\end{verbatim}
A default protocol is provided by the Kernel meta-class {\tt Classifier}:
\begin{verbatim}
context Classifier
  @Operation sendInstance(element,name,args)
    // Get all the operations of element with the 
    // correct name and arity. Select the most 
    // specific and invoke it.
    let arity = args->size then
        operations = element.of().allOperations()
          ->asSeq
          ->select(o | o.name = name and o.arity() = arity)
    in if operations->isEmpty
       then element.error("Cannot handle " + message + "/" + arity)
       else operations->head.invoke(element,args)
       end
    end
  end
\end{verbatim}
Since {\tt Class} is a sub-class of {\tt Classifier}, any sub-class of {\tt Class} that defines a new
{\tt sendInstance} operation will provide a specialized message passing protocol for the instances
of its instances. This can be used to implement specialized operation lookup mechanisms, to 
facilitiate debugging information and to change the basic message passing mechanisms (for example
by defining a class of objects with message queues).

\section{Object Creation}

Objects are created by sending a {\tt new} message to a class together with some initialization
data. The preferred way of invoking the {\tt new} operation of a class is to apply the class
as an operator to the initialization arguments. This is preferred because it is succinct and
because the compiler and XMF VM can handle class instantiation more efficiently in this
form:
\begin{verbatim}
context Classifier
  @Operation invoke(target:Element,args:Seq(Element)):Element
    self.new(args)
  end
\end{verbatim}
The {\tt new} operation is defined by the meta-class of the receiving class. It constitutes
the instantiation procotol for a collection of classes. The class {\tt Classifier} defines
the default instantiation protocol:
\begin{verbatim}
context Classifier
  @Operation new(args:Seq(Element)):Element
    self.new().init(args)
  end
\end{verbatim}
where the operation {\tt new} creates an empty new instance of the receiver and {\tt init}
initializes the new instance. 

Sub-classes of {\tt Classifier} can define their own instantiation protocol. Typically
this will use {\tt super} to create an instance using the default protocol and then perform
some extra computation to initialize the new instance; however, in principle the instantiation
protocol can by-pass the default protocol altogether. To create a raw instance of a class
{\tt C} and add a single slot named {\tt "x"} with initial value {\tt 10} you can do the following:
\begin{verbatim}
let o = Kernel_mkObj()
in Kernel_setOf(o,C);
   Kernel_addAtt(o,Symbol("x",100));
   o
end
\end{verbatim}
Using the kernel-level operations, you can create a completely bespoke instantiation protocol.

\section{Slot Access}

Objects have internal storage in the form of named {\em slots}. Access to a slot value is via the
object and the name of the slot. Slots are named using symbols. Access is defined by the object's 
{\em slot access protocol}. The slot access protocol is used when an expression of the form {\tt o.a}
is performed. Access involves checking that the slot exists and then accessing the value of
the slot.

The existence of a slot can be checked using the {\tt hasSlot} operation
defined by {\tt Object}:
\begin{verbatim}
context Object 
  @Operation hasSlot(name):Boolean
    self.of().hasInstanceSlot(self,name)
  end
\end{verbatim}
The {\tt hasSlot} operation invokes the {\tt hasInstanceSlot} operation of the object's class.
{\tt hasInstanceSlot} forms part of the slot access protocol for the object; the operation
is defined by the object's meta-class. The default definition is provided by {\tt Class} and uses
the kernel operation {\tt Kernel\_hasSlot} to directly check whether there is a machine-level
slot:
\begin{verbatim}
context Class
  @Operation hasInstanceSlot(object,name)
    Kernel_hasSlot(object,name)
  end
\end{verbatim}
Access to a slot's value is provided by the operation {\tt get} defined by {\tt Object}:
\begin{verbatim}
context Object
  @Operation get(name:String):Element
    self.of().getInstanceSlot(self,name)
  end
\end{verbatim}
The operation {\tt getInstanceSlot} is defined by an object's meta-class and describes how to access
the storage associated with an object and a slot name. The default protocol is provided by {\tt Class}
and uses the kernel operation {\tt Kernel\_getSlotValue} to access the machine-level slot (as added
using {\tt Kernel\_addAtt}):
\begin{verbatim}
context Class
  @Operation getInstanceSlot(object,name)
    Kernel_getSlotValue(object,name)
  end
\end{verbatim}
Typically a new slot access protocol is required because a collection of classes implement object
storage in a non-standard way (for example using a table, in a data base or distributed over a network).

\section{Slot Update}

The value of an object's slot can be updated using the object's {\em slot update protocol}. A slot
is updated when an expression of the form {\tt o.a := e} is performed. The class {\tt Object}
provides an operation used to set the value of a slot:
\begin{verbatim}
context Object
  @Operation set(name:String,value:Element):Element
    self.of().setInstanceSlot(self,name,value);
    self
  end
\end{verbatim}
The object's meta-class defines an operation {\tt setInstanceSlot} that forms the update
protocol. The default update protocol is defined by {\tt Class} and uses the kernel-level
operation {\tt Kernel\_setSlotValue} to update the machine-level slot and to invoke any daemons
that are defined on the object:
\begin{verbatim}
context Class
  @Operation setInstanceSlot(object,name,value)
    Kernel_setSlotValue(object,name,value)
  end
\end{verbatim}
A new slot update protocol is used to circumvent the default storage. For example the storage for
a slot may be in a database or accessed over a network. 

\section{Default Parents}

A class is created as an instance of a {\em meta-class}. When a class is created it must have some parents.
The meta-class defines an operation {\tt defaultParents} that produces a set of classes that are the
default parents for its instances. The basic definition for {\tt defaultParents} is provided by
{\tt Classifier}:
\begin{verbatim}
context Classifier
  @Operation defaultParents():Set(Classifier)
    Set{Element}
  end
\end{verbatim}
Most classes are instances of the class {\tt Class}, that overrides the definition as follows:
\begin{verbatim}
context Class
  @Operation defaultParents():Set(Classifier)
    Set{Object}
  end
\end{verbatim}

\section{Example}

Suppose that we want to define a class of objects that can have standard attribute defined 
slots in addition to {\em dynamic slots}. Attribute defined slots are defined at the class
level. Dynamic slots are defined at the object level and can be added and removed 
dynamically. Both types of slot can be accessed and updated via the standard protocols using
{\tt o.a} and {\tt o.a := e} expressions.

In order to implement these objects we require a new slot access and update protocol. The
protocol is defined at the meta-level and is to be called the {\em Elastic} protocol. We require 
two new classes: {\tt ElasticObject} that is the super-class of all user-defined elastic
classes; and, {\tt ElasticClass} that defines the elastic protocol.

The class {\tt ElasticObject} uses a table to contain the dynamic slots:
\begin{verbatim}
context Root
  @Class ElasticObject
    @Attribute slots : Table = Table(100) end
  end
\end{verbatim}
An elastic object provides operations to add and remove the dynamic slots:
\begin{verbatim}
context ElasticObject
  @Operation addSlot(name:String,value)
    slots.put(Symbol(name),value)
  end
\end{verbatim}
\begin{verbatim}
context ElasticObject
  @Operation removeSlot(name)
    slots.remove(Symbol(name))
  end
\end{verbatim}
An elastic object can remove all the dynamic slots:
\begin{verbatim}
context ElasticObject
  @Operation removeAll()
    @For key inTableKeys slots do
      self.removeSlot(key.toString())
    end
  end
\end{verbatim}
An elastic object can increment the values of all the dynamic slots. Note that {\tt incAll}
uses the slot access and update protocol for the object to change the value of the dynamic
slots:
\begin{verbatim}
context ElasticObject
  @Operation incAll()
    @For key inTableKeys slots do
      self.set(key,self.get(key) + 1)
    end
  end
\end{verbatim}
The class {\tt ElasticClass} defines the elastic MOP:
\begin{verbatim}
context Root
  @Class ElasticClass extends Class
  end
\end{verbatim}
{\tt ElsaticObject} must be a parent of any elastic class:
\begin{verbatim}
context ElasticClass
  @Operation defaultParents()
    Set{ElasticObject}
  end
\end{verbatim}
The elastic slot access protocol inspects the {\tt slots} table to see if
the required slot is defined there. If not then the protocol uses {\tt super}
to revert to the default protocol inherited from {\tt Class}:
\begin{verbatim}
context ElasticClass
  @Operation getInstanceSlot(object,name)
    if Kernel_getSlotValue(object,Symbol("slots")).hasKey(name)
    then Kernel_getSlotValue(object,Symbol("slots")).get(name)
    else super(object,name)
    end
  end
\end{verbatim}
\begin{verbatim}
context ElasticClass
  @Operation setInstanceSlot(object,name,value)
    if Kernel_getSlotValue(object,Symbol("slots")).hasKey(name)
    then Kernel_getSlotValue(object,Symbol("slots")).put(name,value)
    else super(object,name,value)
    end
  end
\end{verbatim}
\begin{verbatim}
context ElasticClass
  @Operation hasInstanceSlot(object,name)
    if Kernel_getSlotValue(object,Symbol("slots")).hasKey(name)
    then true
    else super(object,name)
    end
  end
\end{verbatim}
There is no specific need for {\tt sendInstance} in the elastic protocol,
however it is defined for completeness and simply prints a message before
reverting to the default protocol:
\begin{verbatim}
context ElasticClass
  @Operation sendInstance(element,message,args)
    format(stdout,"Sending message ~S(~{,~;~S~})~%",Seq{message,args});
    super(element,message,args)
  end 
\end{verbatim}
There is no specific need for {\tt new} in the elastic protocol,
however it is defined for completeness and simply prints a message before
reverting to the default protocol:
\begin{verbatim}
context ElasticClass
  @Operation new(args)
    format(stdout,"Creating a new instance of an elastic class~%");
    super(args)
  end
\end{verbatim}
The following is an example class definition that specifies its meta-class as
{\tt ElasticClass}:
\begin{verbatim}
context Root
  @Class C metaclass ElasticClass
    @Attribute s : Element end
    @Operation test()
      self.addSlot("x",100);
      self.addSlot("y",200);
      self.addSlot("z",300);
      self.x := self.x + 1;
      self.y := self.y + 1;
      self.z := self.z + 1;
      self.incAll();
      self.s := self.x + self.y + self.z;
      self.removeAll();
      s
    end
  end
\end{verbatim}

\section{Client Connections with a MOP}

XMF provides extensive functionality for constructing and managing internal
models. Data can be constructed and then saved in a number of pre-defined
formats (for example the {\tt xar} format or XML). However, in some cases
the data to be processed by XMF does not originate within XMF or must be
constructed and manipulated by co-ordinating with an external system. For
example, legacy data is often available in standard database formats where
a database engine is used to manage access and update by many concurrent users.
A MOP is an ideal way of hiding away the detail of {\em how} data is represented
and {\em where} the data is maintained. 

This section provides a simple example of how XMF can be used in {\em client mode} 
where object data is provided by a server. The example server we use is very 
simple, but is representative of a wide range of data servers that occur in
commercial applications. A MOP allows us to hide away the details of connecting to
the server and to hide data access and update requests. Once defined, a MOP
allows objects to behave as though they were native XMF objects even though
their representation is managed by an external server. 

The example takes the form of the following components:
\begin{itemize}
\item A server that manages the data. In most cases this will be provided,
in this example the server is defined to manage simple records with integer 
fields. The server will have a communication protocol, in this case the
protocol allows new records to be created and fields to be accessed and
updated. The example server does not implement persistent storage, however
this is easily added, and would usually be provided as part of the
server (as would be the case in a database server).
\item An XMF client that connects to the server and manages the communications
protocol from XMF. The client will be used by each XMF object to communicate with
the server in order to manage its data.
\item An XMF MOP defined as a meta-class that uses an XMF client to manage the
data for instances of its classes. The MOP must arrange for the approprate
instantiation, slot access and slot update operations to be called.
\end{itemize}

\subsection{The Server}

The server is defined as a Java class that is supplied with the port number as
a command line argument, binds a server to the port and waits for a client
connection. Once the client is connected, the server enters a command loop that
reads and processes the commands sent by the client. The cod for the server is
given below (slightly simplified):

\begin{verbatim}
public class Server {

    // Commands performed by the server:

    public static final int NEW = 1;
    public static final int GET = 2;
    public static final int SET = 3;
    
    // Communication channels:

    InputStream in;
    PrintWriter out;
    
    // New objects on the server side are added to the stack.
    // Their ids on the XMF side are the stack positions.

    Stack objects = new Stack();
    
    // A standard implementation of 'bind' for a server:

    public void bind(int port) {
      // Bind to the supplied port.
    }
    
    // A standard implementation of 'accept' for a server:

    public void accept() {
      // Wait and accept a connection from a client.
      // Bind in and out when the client connects.
    }
    
    // The following command loop reads commands and
    // dispatches to appropriate handlers.

    public void loop() {
      while (true) {
        int command = in.read();
        switch (command) {
          case NEW:
            newCommand();
            break;
          case GET:
            getCommand();
            break;
          case SET:
            setCommand();
            break;
        }
      }
    }
    
    // Command arguments are separatd by '!' and terminated by '.'.
    // The following argument readers return values of different
    // types.

    public String readString() {
      // Read and return the next '.' or '!' terminated string.
    }
    
    public int readInt() {
      // Read and return the next '.' or '!' terminated int.
    }
    
    // The GET command is handled by the following method:

    public void getCommand() {
      int id = readInt();
      String slot = readString();
      Record r = (Record) objects.elementAt(id);
      out.write(r.slotValue(slot));
      out.flush();
    }
    
    // The SET command is handled by the following method:

    public void setCommand() {
      int id = readInt();
      String slot = readString();
      int value = readInt();
      Record r = (Record) objects.elementAt(id);
      r.setSlotValue(slot, value);
    }
    
    // The NEW command is handled by the following method:

    public void newCommand() {
      String name = readString();
      Record r = new Record(name);
      int id = objects.size();
      objects.push(r);
      out.write(id);
      out.flush();
    }
    
    // The following method shows how the server is initialized
    // before the command loop is started:

    public static void main(String[] args) {
      int port = Integer.parseInt(args[0]);
      Server server = new Server();
      server.bind(port);
      server.accept();
      server.loop();
    }
}
\end{verbatim}
The Java class {\tt Record} has a name field that is the type of the record and has a 
collection of named slots each of which is accessed using {\tt slotValue} and updated
using {\tt setSlotValue}.

\subsection{The Client}

XMF provides a class {\tt Comms::Client} whose instances can connect to external servers via
ports. A client has an input and output channel that is used to communicate with the server.
Typically an XMF client will implement a communication protocol that is understood by the 
server. The following class is an extension of {\tt Comms::Client} that implements a 
simple protocol for creating records on the server and then accessing and updating the
slots of the records:
\begin{verbatim}
@Class CommandClient extends Client

  // When a command client is created, it immediately tries to 
  // connect to the server. The first argument is the host 
  // ("" means local machine) and the second argument is the 
  // port number.
  
  @Constructor()
    self.connect("",99)
  end
  
  // An operation for sending commands. The commands are 
  // encoded as integers and the args are supplied as a 
  // sequence of strings and integers. The protocol separates 
  // args with ! and terminates the command with .
  
  @Operation writeCommand(command:Integer . args:Seq(String))
    output.write(command);
    @For arg in args do
      output.write(arg);
      if not isLast then format(output,"!") end
    end;
    output.write(".");
    output.flush()
  end
  
  // An operation for receiving data from the server. Note that 
  // if no data has been sent then the input channel *may* 
  // produce -1. In this case we just want to block until some 
  // real data is sent.
  
  @Operation readData()
    let value = 0-1
    in @While value < 0 do
         value := input.read()
       end;
       value
    end
  end
  
  // A NEW command is the constant 1. The server will send the 
  // unique id for the new instance of Record.
  
  @Operation newCommand(name:String):Element
    self.writeCommand(1,name);
    self.readData()
  end
  
  // A GET command is the constant 2. We send the id of the 
  // record and then read the slot value. In this simple 
  // example slot values can only be a  single byte of data 
  // in the range 0 - 255.
  
  @Operation getCommand(id:Integer,name:String)
    self.writeCommand(2,id,name);
    self.readData()
  end
  
  // A SET command is the constant 3. We don't need to wait 
  // for a result.
  
  @Operation setCommand(id:Integer,name:String,value:Integer)
    self.writeCommand(3,id,name,value)
  end
end
\end{verbatim}

\subsection{The MOP}

We have defined a Java server class and an XMF client class. If we start the server
and create an instance of {\tt CommandClient} then the client connects to the server
and can communicate with it. We can use this connection explicitly to implement 
objects, however XMF facitlities that have been defined without knowledge of 
{\tt CommandClient} must all up updated to make use of the connection. In general,
this is unacceptable, we need a mechanism that will allow new objects whose representation
is in the server to be seamlessly integrated with the existing XMF technologies.

A MOP is a collection of operation definitions that allows a class to implement 
its own instance creation and slot management mechanisms. The operations are integrated
within the XMF VM so that the new definitions are used whenever instances of
a class are referenced at the VM level. The upper levels of XMF are completely
unaware of the hi-jacking going on at the lower levels. A MOP is a way of integrating
completely new object representations with existing mechanisms without changing the
mechanisms in any way.

To create a MOP for our command client we need two classes. The first, {\tt ExternalObject}
is the super-class of any class whose instances are to be managed by the server. The
class {\tt ExternalObject} defines all the representation and operations that are necessary
to support an external object on the client side. In the case of command client, we need
to retain the unique id of the object:
\begin{verbatim}
context Root
  @Class ExternalObject
    @Attribute id : Integer end
    @Constructor(id) ! end
  end
\end{verbatim}
An external class has instances that are external objects. A MOP is implemented by defining a
meta-class that optionally specifies the default parents for its instances, specifies the
behaviour of {\tt new} for its instances and defines a slot access protocol and a message passing
protocol for the objects that it manages (i.e. the instances of its instances). The MOP
is implemented by {\tt ExternalClass} defined below, specific comments are in-line: 
\begin{verbatim}
context Root
  @Class ExternalClass extends Class
    
    // An external class uses a command client to communicate
    // with the server.
    
    @Attribute client : CommandClient (!) end
    
    // The super-classes of an external class must include the
    // clas ExternalObject:
    
    @Operation defaultParents()
      Set{ExternalObject}
    end
    
    // To get the slot of an external object we use Kernel_getSlotValue
    // when the slot is 'id' (since this is on the client side) otherwise
    // use the client object to communicate with the server to get the
    // slot value.
    
    @Operation getInstanceSlot(object,name)
      if name.toString() = "id"
      then Kernel_getSlotValue(object,name.asSymbol())
      else client.getCommand(object.id,name.toString())
      end
    end
    
    // The server allows a record to have any number of slots:
    
    @Operation hasInstanceSlot(object,name)
      true
    end
    
    // To set the slo of an external object we user Kernel_setSlotValue
    // when the slot is on the client side (id) otherwise use the client 
    // object to communicate with the server to update the slot value.
    
    @Operation setInstanceSlot(object,name,value)
      if name.toString() = "id"
      then Kernel_setSlotValue(object,name.asSymbol(),value)
      else client.setCommand(object.id,name.toString(),value)
      end
    end
    
    // To create an instance of an external class, first create a standard
    // instance (this represents the client view of the new object) then
    // request the server to create a record.
    
    @Operation new(args:Seq(Element))
      let o = super(args)
      in o.id := client.newCommand(self.name.toString())
      end
    end
  end
\end{verbatim}

\end{document}